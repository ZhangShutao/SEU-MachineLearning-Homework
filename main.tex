\documentclass[10pt]{beamer}
\usepackage{xeCJK}
\usepackage{listings}
\usepackage{tikz}
\usetikzlibrary{arrows,automata}

\usetheme[
%	sidebar, % 默认不显示包含幻灯片结构的边框。如设置sidebar选项,则参考AAU模板显示左边框
	footline,
	blue, % 主色调默认为红色,色调可以选择red和blue
%	wide, % 幻灯片的长宽比默认为4:3,如设置了wide选项则为16:9
	hideallsubsections, % 默认显示所有等级的标题。如设置了hideallsubsections,
	                    % 则不显示小节标题
	mathserif, % 默认公式字体是钝化的,如设置mathserif选项则采用正常的公式字体
%	english, % 默认幻灯片环境为中文,如设置english选项则采用英文的章节和图表编号
	sectiontoc, % 设置sectiontoc选项则在每节(section)之前添加一个所有节的目录,
	            % 并标明本节在整个幻灯片中的位置,不建议和\part层级一同使用
]{SEUstyle}

\title[第四组报告]{强化学习在FPS游戏中的应用 \\ 强化学习与规划}
\author[张舒韬 et al.]{刘翔~吴江恒~张舒韬~赵倩隆}
\institute[SEU CS Dept. Team 4]{东南大学\ 计算机科学与工程学院\ 第4组}

\begin{document}
	{\background
		\begin{frame}[plain,noframenumbering]
			\titlepage
		\end{frame}
	}

	\begin{frame}{总目录}
		第I部分 强化学习与规划
		\tableofcontents[part=1]
		第II部分 强化学习在FPS游戏中的应用
		\tableofcontents[part=2]
	\end{frame}

	\part{强化学习与规划}\label{part:rl-and-planning}
	
	\part{强化学习在FPS游戏中的应用}\label{part:rl-in-fps}
	
	{\background%末页致谢
		\begin{frame}[plain,noframenumbering]
			\finalpage{{\huge 感谢观看!\\ \small Q \& A}}
		\end{frame}
	}
	
\end{document}